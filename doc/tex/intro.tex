\section{Introduction}
\label{sec:introduction}

The \verb|paws| software package aims to provide 
a fast and lean platform for building and executiong workflows for data processing.
It was originally developed to perform analysis of diffraction images 
for ongoing projects at SLAC/SSRL.
At the core of \verb|paws| is a workflow engine
that uses a library of simple operations
to perform data analysis, for single inputs, 
batches of inputs, or inputs generated in real time. 

\verb|paws| is currently written in Python,
making use of the PySide Qt library bindings
for GUI elements as well as data structures and threading facilities.
Internally, \verb|paws| keeps track of data in Qt-based tree structures,
and much of the graphical part of the program involves 
interacting with these trees through the Qt model-view framework.

\verb|paws| also provides its functionality to \verb|xi-cam|,
a general-purpose synchrotron x-ray diffraction data analysis package
with lots of fast and user-friendly data analysis tools.
TODO: Cite and link to xi-cam here.

Some the core goals of \verb|paws|:
\begin{itemize}
\item Eliminate redundant development efforts 
\item Streamline and standardize routine data analysis
\item Simplify data storage and provide large-scale analysis 
\item Perform data analysis in real time for results-driven feedback
\end{itemize}

The \verb|paws| developers would love to hear from you
if you have wisdom, thoughts, haikus, bugs, artwork, or suggestions.
Limericks are also welcome.
Get in touch with us at \verb|paws-developers@slac.stanford.edu|.


