\section{Introduction}
\label{sec:introduction}

The \verb|slacx| software package aims to provide 
a fast and lean platform for exploring and processing image-like data.
It was originally developed to perform analysis of x-ray diffraction patterns 
for ongoing projects at SLAC/SSRL.
At the core of \verb|slacx| is a workflow engine
that uses a library of simple operations
to perform data analysis, for single inputs, 
batches of inputs, or inputs generated in real time. 

The library of operations built into \verb|slacx|
was designed with specific workflows in mind,
and in some cases an effort was made to provide more general functionality.
Some operations have been used a lot and are known to be stable,
while others may introduce bugs if used in ways other than intended.
In order to limit the required libraries to a few core modules,
\verb|slacx| defaults to a state where all operations are unavailable to the user.
The user must choose which operations will be used for their workflow,
and these will be imported as they are enabled.
The list of enabled operations is saved in a configuration file
so that they do not need to be enabled every time \verb|slacx| is run.

\verb|slacx| is currently written in Python,
making use of the PySide Qt library bindings
for GUI elements as well as data structures and threading facilities.
Internally, \verb|slacx| keeps track of data in a Qt-based tree,
and much of the graphical part of the program involves 
interacting with these trees,
as implemented in a Qt model-view framework.

\verb|slacx| also contains a plugin that provides its functionality to \verb|xi-cam|,
a general-purpose synchrotron diffraction analysis software package
with lots of fast and easy data analysis tools
develped by Pandolfi, et al at Lawrence Berkeley National Lab and the Camera Institute.
NB: Cite and link to xi-cam here.

Some the core goals of \verb|slacx|:
\begin{itemize}
\item Eliminate redundant development efforts 
\item Streamline and standardize routine data analysis
\item Simplify data storage and provide large-scale analysis 
\end{itemize}

The \verb|slacx| developers would love to hear from you
if you have wisdom, thoughts, haikus, bugs, artwork, suggestions, or limericks.
Get in touch with us at \verb|slacx-developers@slac.stanford.edu|.




